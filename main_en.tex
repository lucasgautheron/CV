% !TEX TS-program = lualatex

\AtBeginDocument{\hypersetup{pdfborder = 0 1 1, linkcolor=blue}}
\documentclass[11pt,a4paper]{moderncv}
\moderncvtheme[blue]{classic}
\AtBeginDocument{\recomputelengths}
\usepackage[utf8]{inputenc}
\usepackage[top=1.1cm, bottom=1.1cm, left=2cm, right=2cm]{geometry}
\setlength{\hintscolumnwidth}{2.5cm}

\RequirePackage{filecontents}
\usepackage[backend=bibtex,maxbibnames=99,sorting=ydnt,giveninits=true]{biblatex}

\firstname{Lucas}
\familyname{Gautheron}
%\title{Etudiant en Physique}              
% \address{242 Rue de Tolbiac}{75013 Paris}    
\email{lucas.gautheron@gmail.com}                      
\homepage{lucasgautheron.github.io}
\mobile{06 79 23 86 47} 
\extrainfo{29/01/1995}
% \photo[70pt][0.5pt]{photo.png}

\def\makenamesetup{%
  \def\bibnamedelima{~}%
  \def\bibnamedelimb{ }%
  \def\bibnamedelimc{ }%
  \def\bibnamedelimd{ }%
  \def\bibnamedelimi{ }%
  \def\bibinitperiod{.}%
  \def\bibinitdelim{~}%
  \def\bibinithyphendelim{.-}}    
\newcommand*{\makename}[2]{\begingroup\makenamesetup\xdef#1{#2}\endgroup}

\newcommand*{\boldname}[3]{%
  \def\lastname{#1}%
  \def\firstname{#2}%
  \def\firstinit{#3}}
\boldname{}{}{}

% Patch new definitions
\renewcommand{\mkbibnamegiven}[1]{%
  \ifboolexpr{ ( test {\ifdefequal{\firstname}{\namepartgiven}} or test {\ifdefequal{\firstinit}{\namepartgiven}} ) and test {\ifdefequal{\lastname}{\namepartfamily}} }
  {\mkbibbold{#1}}{#1}%
}

\renewcommand{\mkbibnamefamily}[1]{%
  \ifboolexpr{ ( test {\ifdefequal{\firstname}{\namepartgiven}} or test {\ifdefequal{\firstinit}{\namepartgiven}} ) and test {\ifdefequal{\lastname}{\namepartfamily}} }
  {\mkbibbold{#1}}{#1}%
}

\addbibresource{publications.bib}
\addbibresource{talks.bib}

\boldname{Gautheron}{Lucas}{L.}


\begin{document}
\maketitle

\section{Education}
\cventry{2022-2025}{PhD student}{University of Wuppertal}{Wuppertal (Germany)}{Dynamics of a research program in high-energy physics: the case of supersymmetry (Research Training Group 2696, ``Transformations of science and technology since 1800: topics, processes, institutions
'') [\textit{on-going}]}{}
\cventry{Feb 2023--\\Apr 2023}{Visiting PhD student}{Medialab, Sciences Po Paris}{Paris}{}{}
\cventry{2021-2022}{M.A}{Université de Paris}{Paris}{History and Philosophy of Science [Class rank: 1st]}{Thesis : ``Too beautiful to be false, or too beautiful to be true? Searching supersymmetry at the \textit{Large Hadron Collider}''. Supervisors: Olivier Darrigol, Elisa Omodei.} 
\cventry{2014-2018}{Master 1}{Ecole Normale Supérieure de Cachan}{Cachan}{Fundamental Physics}{Full scholarship. Experimental and theoretical physics, mathematics. Options: Symmetries and path integrals; Astrophyics and Cosmology.}
\cventry{2012-2014}{PCSI/PC*}{Lycée Berthollet}{Annecy}{}{``Classe préparatoire aux grandes écoles''}
% \cventry{2009 -- 2012}{Baccalauréat S option SI}{Lycée Louis Lachenal}{Argonay (74)}{}{}

\section{Research}

\cventry{Sep 2020 to Nov 2021}{Engineer}{Laboratoire de Sciences Cognitives et Psycholinguistique (LSCP - DEC - École Normale Supérieure)}{}{Paris}{
Study of language acquisition across cultures through naturalistic long-form audio recordings. Supervised by Alejandrina Cristia.
\begin{itemize}%
\item Creation of a python package for the management, storage and analysis of large datasets ($\mathcal{O}(10^4)$ hours of audio)
\item Signal processing on long-form audio
\item Statistical analysis (bayesian inference)
\item CNRS Training "Basics of Machine Learning and Deep Learning" (28h)
\end{itemize}
}

\cventry{Oct 2016 -- Jan 2017}{Research Internship}{Laboratoire Univers et Théories (LUTH - INSU - CNRS)}{}{Paris Meudon}{
Influence of the nuclei distribution on the electron capture rates and neutrino scattering in core-collapse supernovae.  Supervised by Micaela Oertel.
\begin{itemize}%
\item Implementation of electron capture rates and neutrino scattering cross-sections calculations in a core-collapse supernova simulation code (Fortran, C++).
\end{itemize}
}

\cventry{May 2016 -- Jul 2016}{Research Internship}{Laboratoire de Physique Nucléaire et des Hautes Énergies (LPNHE - IN2P3 - CNRS)}{}{Paris}{
Diphoton analysis and phenomenology for the ATLAS experiment. Supervised by Lydia Roos.
\begin{itemize}%
\item New parameterization of the diphoton invariant mass distribution for a spin-2 decaying particle signal (Pythia, ROOT, RooFit, C++, Python, FeynRules, CalcHEP)
\item NLO corrections for the spin-2 signal (MadGraph5\_aMC\_atNLO)
\item Signal-background interferences
\end{itemize}
}

\cventry{Oct 2015 -- Jan 2016}{Research Internship}{Laboratoire d'Annecy-Le-Vieux de Physique Théorique (LAPth - IN2P3 - CNRS)}{}{Annecy-Le-Vieux}{
Cosmology. Supervised par Richard Taillet.
\begin{itemize}%
\item Creation of an Internet website for students about the history of modern cosmology (\href{http://cosmology.education/}{http://cosmology.education/})
\item Development of several simulations in C++ to illustrate the website.
\end{itemize}
}

\cventry{May 2015 -- Jul 2015}{Research Internship}{Laboratoire d'Annecy-Le-Vieux des Particules (LAPP - IN2P3 - CNRS)}{}{Annecy-Le-Vieux}{
Particle physics for the ATLAS experiment. Supervised by Stéphane Jézéquel.
\begin{itemize}%
\item Diphoton events analysis and local/global significance calculations with ROOT.
\item Analysis of the performance of a new tracker prototype for HL-LHC, using MC simulations
\item Development of a simulation to assess the impact of thermal radiation over the temperature of parts of the tracker, as part of the design of a cooling system upgrade
\item Development of a simulation to calculate the intersections of charged particles with the sensors of a tracker prototype
\end{itemize}
}

\nocite{*} % <==========================================================

\printbibliography[title={Publications},notkeyword=talks]
\printbibliography[title={Talks},keyword=talks]


\section{Journalism}

%\cventry{Novembre 2020 à Aujourd'hui}{Pigiste}{Le Média}{}{Montreuil}{Recensions d'ouvrages.}

\cventry{Dec 2019 to Nov 2020}{President}{Société de Production Le Média}{}{Montreuil}{Management of a production company with more than 12 full-time equivalent workers. Publication Manager.
Marketing procedures optimization, development of revenue forecasting models and audience data analysis.}

\cventry{Septemter 2018 to Sep 2020}{Journalist}{Le Média}{}{Montreuil}{Specializing in data journalism and book reviews \url{https://www.lemediatv.fr/auteurs/lucas-gautheron-9DAnWoo5Tlav1trWgg_Qlw/articles}.
}

% \cventry{Sep 2018 -- Augst 2019}{Web editor}{Le Média}{}{Montreuil}{Editorial work, Audience analysis, Post-production (editing)}

\section{Development}

\cventry{Juillet 2013}{Developer}{Électricité réseau Distribution de France (ErDF)}{Annecy}{}{
\begin{itemize}%
\item Design of an archive database and search system (PHP/MySQL)
\item Automated retrieval of large amounts of data from other company-wide applications. 
\end{itemize}}
\cventry{Mars 2011 à 2014}{Developer}{AssaultCube}{}{}{Development of a C++ 3D video game as part of an international team of volunteers}

\section{Skills}

\subsection{Computer}
\cvitem{Programming}{Python, C, C++, Fortran}
\cvitem{Scientific software}{numpy, scipy, scikit-learn, MadGraph5\_aMC\_@NLO, Pythia, ROOT/RooFit, stan}
\cvitem{Data}{Pandas, MySQL, HDF}
\cvitem{Web}{PHP, HTML, JS, CSS}
\cvitem{Video}{Adobe Premiere Pro}

\subsection{Languages}
\cvlanguage{English}{Toefl IBT: 104}{}
\cvlanguage{French}{Native}{}
\cvlanguage{Spanish}{Beginner}{}

\end{document}