
\AtBeginDocument{\hypersetup{pdfborder = 0 1 1, linkcolor=blue}}
\documentclass[11pt,a4paper]{moderncv}
\moderncvtheme[blue]{classic}
\AtBeginDocument{\recomputelengths}
\usepackage[utf8]{inputenc}
\usepackage[top=1.1cm, bottom=1.1cm, left=2cm, right=2cm]{geometry}
\setlength{\hintscolumnwidth}{2.5cm}

\RequirePackage{filecontents}
\usepackage[backend=biber,maxbibnames=99,defernumbers=true,sorting=ydnt,giveninits=true]{biblatex}

\firstname{Lucas}
\familyname{Gautheron}
%\title{Etudiant en Physique}              
\address{3 Rue Bellier-Dedouvre}{75013 Paris}    
\email{lucas.gautheron@gmail.com}                      
\homepage{sciencestechniques.fr}
\mobile{06 79 23 86 47} 
\extrainfo{29/01/1995}
\photo[70pt][0.5pt]{photo.png}

\begin{filecontents*}{\jobname.bib}

@inproceedings{gautheron:hal-03047153,
  TITLE = {{Longform recordings : Opportunities and challenges}},
  AUTHOR = {Gautheron, Lucas and Lavechin, Marvin and Riad, Rachid and Scaff, Camila and Cristia, Alejandrina},
  URL = {https://hal.archives-ouvertes.fr/hal-03047153},
  BOOKTITLE = {{LIFT 2020 - 2{\`e}mes journ{\'e}es scientifiques du Groupement de Recherche ''Linguistique informatique, formelle et de terrain''}},
  ADDRESS = {Montrouge / Virtual, France},
  EDITOR = {Poibeau, Thierry and Parmentier, Yannick and Schang, Emmanuel},
  PUBLISHER = {{CNRS}},
  PAGES = {64-71},
  YEAR = {2020},
  MONTH = Dec,
  KEYWORDS = {daylong recordings ; ecological validity ; automatic speech processing},
  HAL_ID = {hal-03047153},
  HAL_VERSION = {v1},
}

@article{Lavechin2021,
  doi = {10.31234/osf.io/pt9xq},
  url = {https://doi.org/10.31234/osf.io/pt9xq},
  note = {\url{https://doi.org/10.31234/osf.io/pt9xq}},
  year = {2021},
  month = mar,
  publisher = {Center for Open Science},
  author = {Marvin Lavechin and Maureen de Seyssel and Lucas Gautheron and Emmanuel Dupoux and Alejandrina Cristia},
  title = {Reverse-engineering language acquisition with child-centered long-form recordings},
  journal = {Submitted to \emph{Annual Reviews of Linguistics}}
}

@article{gautheron_rochat_cristia_2021,
 title={Managing, storing, and sharing long-form recordings and their annotations},
 url={https://psyarxiv.com/w8trm},
 note = {\url{https://psyarxiv.com/w8trm}},
 DOI={10.31234/osf.io/w8trm},
 publisher={PsyArXiv},
 author={Gautheron, Lucas and Rochat, Nicolas and Cristia, Alejandrina},
 year={2021},
 month={May},
 journal={Submitted to \emph{Language Resources and Evaluation}}
}
\end{filecontents*}


\def\makenamesetup{%
  \def\bibnamedelima{~}%
  \def\bibnamedelimb{ }%
  \def\bibnamedelimc{ }%
  \def\bibnamedelimd{ }%
  \def\bibnamedelimi{ }%
  \def\bibinitperiod{.}%
  \def\bibinitdelim{~}%
  \def\bibinithyphendelim{.-}}    
\newcommand*{\makename}[2]{\begingroup\makenamesetup\xdef#1{#2}\endgroup}

\newcommand*{\boldname}[3]{%
  \def\lastname{#1}%
  \def\firstname{#2}%
  \def\firstinit{#3}}
\boldname{}{}{}

% Patch new definitions
\renewcommand{\mkbibnamegiven}[1]{%
  \ifboolexpr{ ( test {\ifdefequal{\firstname}{\namepartgiven}} or test {\ifdefequal{\firstinit}{\namepartgiven}} ) and test {\ifdefequal{\lastname}{\namepartfamily}} }
  {\mkbibbold{#1}}{#1}%
}

\renewcommand{\mkbibnamefamily}[1]{%
  \ifboolexpr{ ( test {\ifdefequal{\firstname}{\namepartgiven}} or test {\ifdefequal{\firstinit}{\namepartgiven}} ) and test {\ifdefequal{\lastname}{\namepartfamily}} }
  {\mkbibbold{#1}}{#1}%
}

\addbibresource{\jobname.bib}
\boldname{Gautheron}{Lucas}{L.}


\begin{document}
\maketitle

\section{Education}
\cventry{2021-2022}{M.A}{Université de Paris}{Paris (75)}{History and Philosophy of Science [\textit{on-going}]}{}
\cventry{2014-2018}{Master 1}{Ecole Normale Supérieure de Cachan}{Cachan (94)}{Fundamental Physics}{Full scholarship. Experimental and theoretical physics, mathematics. Options: Symmetries and path integrals; Astrophyics and Cosmology.}
\cventry{2012-2014}{PCSI/PC*}{Lycée Berthollet}{Annecy (74)}{}{``Classe préparatoire aux grandes écoles''}
\cventry{2009 -- 2012}{Baccalauréat S option SI}{Lycée Louis Lachenal}{Argonay (74)}{}{}

\section{Research}

\cventry{September 2020 to November 2021}{Engineer}{Laboratoire de Sciences Cognitives et Psycholinguistique (LSCP - DEC - École Normale Supérieure)}{}{Paris}{
Study of language acquisition across cultures through naturalistic long-form audio recordings.
\begin{itemize}%
\item Creation of a python package for the management, storage and analysis of large datasets ($\mathcal{O}(10^4)$ hours of audio)
\item Signal processing on long-form audio
\item Data analysis (bayesian inference)
\item CNRS Training "Basics of Machine Learning and Deep Learning" (28h)
\end{itemize}
}

\cventry{October 2016 -- January 2017}{Research Internship}{Laboratoire Univers et Théories (LUTH - INSU - CNRS)}{}{Paris Meudon}{
Influence of the nuclei distribution on the electron capture rates and neutrino scattering in core-collapse supernovae.  Supervised by Micaela Oertel.
\begin{itemize}%
\item Implementation of electron capture rates and neutrino scattering cross-sections calculations in a core-collapse supernova simulation code (Fortran, C++).
\end{itemize}
}

\cventry{May 2016 -- July 2016}{Research Internship}{Laboratoire de Physique Nucléaire et des Hautes Énergies (LPNHE - IN2P3 - CNRS)}{}{Paris}{
Diphoton analysis and phenomenology for the ATLAS experiment. Supervised by Lydia Roos.
\begin{itemize}%
\item New parameterization of the diphoton invariant mass distribution for a spin-2 decaying particle signal (Pythia, ROOT, RooFit, C++, Python, FeynRules, CalcHEP)
\item NLO corrections for the spin-2 signal (MadGraph5\_aMC\_atNLO)
\item Signal-background interferences
\end{itemize}
}

\cventry{October 2015 -- January 2016}{Research Internship}{Laboratoire d'Annecy-Le-Vieux de Physique Théorique (LAPth - IN2P3 - CNRS)}{}{Annecy-Le-Vieux}{
Cosmology. Supervised par Richard Taillet.
\begin{itemize}%
\item Creation of an Internet website for students about the history of modern cosmology (\href{http://cosmology.education/}{http://cosmology.education/})
\item Development of several simulations in C++ to illustrate the website.
\end{itemize}
}

\cventry{May 2015 -- July 2015}{Research Internship}{Laboratoire d'Annecy-Le-Vieux des Particules (LAPP - IN2P3 - CNRS)}{}{Annecy-Le-Vieux}{
Particle physics for the ATLAS experiment. Supervised by Stéphane Jézéquel.
\begin{itemize}%
\item Diphoton events analysis and local/global significance calculations with ROOT.
\item Analysis of the performance of a new tracker prototype for HL-LHC, using MC simulations
\item Development of a simulation to assess the impact of thermal radiation over the temperature of parts of the tracker, as part of the design of a cooling system upgrade
\item Development of a simulation to calculate the intersections of charged particles with the sensors of a tracker prototype
\end{itemize}
}

\nocite{*} % <==========================================================

\printbibliography[title={Publications}]

\section{Journalism}

%\cventry{Novembre 2020 à Aujourd'hui}{Pigiste}{Le Média}{}{Montreuil}{Recensions d'ouvrages.}

\cventry{December 2019 to November 2020}{President}{Société de Production Le Média}{}{Montreuil}{Management of a production company with more than 12 full-time equivalent workers. Publication Manager.
Marketing procedures optimization, revenue and audience data analysis, development of revenue forecasting models.}

\cventry{September 2018 to September 2020}{Journalist}{Le Média}{}{Montreuil}{Specializing in data visualization and book reviews \url{https://www.lemediatv.fr/auteurs/lucas-gautheron-9DAnWoo5Tlav1trWgg_Qlw/articles}.
}

% \cventry{September 2018 -- August 2019}{Digital Manager}{Le Média}{}{Montreuil}{
% Editorial work, Audience analysis, Post-production (editing)}

\section{Development}

\cventry{Juillet 2013}{Developer}{Électricité réseau Distribution de France (ErDF)}{Annecy}{}{
\begin{itemize}%
\item Design of an archive database and search system (PHP/MySQL)
\item Automated retrieval of large amounts of data from other company-wide applications. 
\end{itemize}}
\cventry{Mars 2011 à 2014}{Developer}{AssaultCube}{}{}{Development of a C++ 3D video game as part of an international team of volunteers}

\section{Skills}

\subsection{Computer}
\cvitem{Programming}{Python, C, C++, Fortran}
\cvitem{Scientific software}{numpy, scipy, scikit-learn, MadGraph5\_aMC\_@NLO, Pythia, ROOT/RooFit, stan}
\cvitem{Data}{Pandas, MySQL, HDF}
\cvitem{Web}{PHP, HTML, JS, CSS}
\cvitem{Video}{Adobe Premiere Pro}

\subsection{Languages}
\cvlanguage{English}{C1-C2}{}
\cvlanguage{French}{}{}
\cvlanguage{Spanish}{Beginner}{}

\end{document}