
\AtBeginDocument{\hypersetup{pdfborder = 0 1 1, linkcolor=blue}}
  
\documentclass[11pt,a4paper]{moderncv}
\moderncvtheme[blue]{classic}
\AtBeginDocument{\recomputelengths}
\usepackage[utf8]{inputenc}
\usepackage[top=1.1cm, bottom=1.1cm, left=2cm, right=2cm]{geometry}
\setlength{\hintscolumnwidth}{2.5cm}

\firstname{Lucas}
\familyname{Gautheron}
%\title{Etudiant en Physique}              
\address{3 Rue Bellier-Dedouvre}{75013 Paris}    
\email{lucas.gautheron@gmail.com}                      
\homepage{sciencestechniques.fr}
\mobile{06 79 23 86 47} 
\extrainfo{29/01/1995}

\RequirePackage{filecontents}
\begin{filecontents*}{\jobname.bib}
@inproceedings{gautheron:hal-03047153,
  TITLE = {{Longform recordings : Opportunities and challenges}},
  AUTHOR = {Gautheron, Lucas and Lavechin, Marvin and Riad, Rachid and Scaff, Camila and Cristia, Alejandrina},
  URL = {https://hal.archives-ouvertes.fr/hal-03047153},
  BOOKTITLE = {{LIFT 2020 - 2{\`e}mes journ{\'e}es scientifiques du Groupement de Recherche ''Linguistique informatique, formelle et de terrain''}},
  ADDRESS = {Montrouge / Virtual, France},
  EDITOR = {Poibeau, Thierry and Parmentier, Yannick and Schang, Emmanuel},
  PUBLISHER = {{CNRS}},
  PAGES = {64-71},
  YEAR = {2020},
  MONTH = Dec,
  KEYWORDS = {daylong recordings ; ecological validity ; automatic speech processing},
  HAL_ID = {hal-03047153},
  HAL_VERSION = {v1},
}
\end{filecontents*}


\begin{document}
\maketitle

\section{Formation}
\cventry{2014-2018}{M1}{Ecole Normale Supérieure de Cachan}{Cachan (94)}{}{Fonctionnaire stagiaire. Physique expérimentale et théorique, mathématiques.}
\cventry{2012-2014}{PCSI/PC*}{Lycée Berthollet}{Annecy (74)}{}{Classe préparatoire aux grandes écoles}
\cventry{2009 -- 2012}{Baccalauréat S option SI}{Lycée Louis Lachenal}{Argonay (74)}{}{Félicitations du Jury}

\section{Recherche}

\cventry{Septembre 2020 -- Mars 2021}{Data Manager}{Laboratoire de Sciences Cognitives et Psycholinguistique (LSCP - DEC - École Normale Supérieure)}{}{Paris}{
Acquisition du langage à travers les cultures.
\begin{itemize}%
\item Développement de process et d'un package python pour la gestion et le stockage de datasets volumineux ($\mathcal{O}(10^4)$ heures d'enregistrements audio + bases de données et annotations)
\item Automatisation des tâches de classification (RNN) sur le cluster (slurm)
\item Extraction de statistiques sur les données (python, R)
\end{itemize}
}

\cventry{Octobre 2016 -- Janvier 2017}{Stage de recherche}{Laboratoire Univers et Théories (LUTH - INSU - CNRS)}{}{Paris Meudon}{
Influence de la distribution complète de noyaux pendant une supernova à effondrement de coeur sur les taux de capture électronique et la diffusion de neutrinos.
\begin{itemize}%
\item Nouveau code pour générer les taux de capture électronique et les sections efficaces de diffusion de neutrinos durant une supernova (C++)
\item Integration de ces résultats dans une simulation du processus de supernova (CoCoNuT, Fortran).
\end{itemize}
}

\cventry{Mai 2016 -- Juillet 2016}{Stage de recherche}{Laboratoire de Physique Nucléaire et des Hautes Énergies (LPNHE - IN2P3 - CNRS)}{}{Paris}{
Analyse diphoton et phénoménologie pour l'expérience ATLAS. (Supervisé par Lydia Roos)
\begin{itemize}%
\item Nouvelle paramétrisation pour un signal de désintégration de spin-2 dans la distribution de masse invariante diphoton, en vue d'ICHEP 2016 (Pythia, ROOT, RooFit, C++, Python, FeynRules, CalcHEP)
\item Corrections NLO pour le signal spin-2 (MadGraph5\_aMC\_atNLO)
\end{itemize}
}

\cventry{Octobre 2015 -- Janvier 2016}{Stage de recherche}{Laboratoire d'Annecy-Le-Vieux de Physique Théorique (LAPth - IN2P3 - CNRS)}{}{Annecy-Le-Vieux}{
Cosmologie, sous la supervision de Richard Taillet.
\begin{itemize}%
\item Création d'une site Internet pédagogique sur l'histoire de la cosmologie moderne, sur la base d'un important travail bibliographique (\href{http://cosmology.education/}{http://cosmology.education/})
\item Développement de plusieurs simulations (en C++) pour l'animation du site
\end{itemize}
}

\cventry{Mai 2015 -- Juillet 2015}{Stage de recherche}{Laboratoire d'Annecy-Le-Vieux des Particules (LAPP - IN2P3 - CNRS)}{}{Annecy-Le-Vieux}{
Physique des particules pour l'expérience ATLAS. Stéphane Jézéquel.
\begin{itemize}%
\item Analyse des événements diphoton et calculs simplifiés de significance locale et globale avec ROOT
\item Anaylse des performances d'une prototype de nouveua tracker en vue d'HL-LHC à l'aide de simulations MC
\item Developpement d'une simulation pour estimer l'impact du rayonnement thermique sur la température d'un module pixel du tracker dans le cadre du design du système de refroidissement
\item Développement d'une simulation calculant les intersections de particules chargées dans le tracker avec les capteurs pixels pour un prototype de tracker
\end{itemize}
}

\nocite{*} % <==========================================================
\bibliographystyle{unsrt}
\bibliography{\jobname} % To use bib file created by filecontents

\section{Médias}

\cventry{Janvier 2020 à Novembre 2020}{Président}{Société de Production Le Média}{}{Montreuil}{Direction de la publication. Optimisation des procédures de démarchage des client. Analyse de données à des fins de ciblage. Développement d'outils de gestion et de prédictions de recettes.}

\cventry{Décembre 2019 à Septembre 2020}{Journaliste (CDI)}{Le Média}{}{Montreuil}{Spécialisation en analyse de données et représentation graphique de données. \href{https://www.lemediatv.fr/auteurs/lucas-gautheron-9DAnWoo5Tlav1trWgg_Qlw/emissions}{Accéder aux contenus :\\} \url{https://www.lemediatv.fr/auteurs/lucas-gautheron-9DAnWoo5Tlav1trWgg_Qlw/emissions}}

\cventry{Septembre 2018 -- Août 2019}{Responsable Numérique}{Le Média}{}{Montreuil}{
\begin{itemize}
\item Gestion de la diffusion des contenus sur les réseaux sociaux : éditorialisation, montage, sous-titrage, programmation.
\item Production (cadrage, réalisation en direct)
\item Post-production (montage)
\end{itemize}
}

\section{Développement}

\cventry{Juillet 2013}{Développeur}{Électricité réseau Distribution de France (ErDF)}{Annecy}{}{Participation au développement d'une application pour l'organisation du travail des salariés
\begin{itemize}%
\item Implémentation d'un système d'archive (PHP/MySQL).
\item Automatisation de l'accès aux données depuis plusieurs applications externes (cURL). \newline{}
\end{itemize}}
\cventry{Mars 2011 à 2014}{Développeur}{AssaultCube}{}{}{Participation au développement d'un jeu vidéo au sein d'une équipe internationale (C++).}

\section{Compétences}

\subsection{Informatique}
\cvitem{Programmation}{Python, C, C++, Fortran}
\cvitem{Logiciels scientifiques}{Gnuplot, Mathematica, numpy, scipy, scikit-learn, MadGraph5\_aMC\_@NLO, Pythia, ROOT/RooFit, stan}
\cvitem{Données}{Pandas, MySQL, HDF}
\cvitem{Web}{PHP, HTML, JS, CSS, XSLT}
\cvitem{Systèmes}{Windows, Linux (Debian/Ubuntu), Mac OS}
\cvitem{Bureautique}{LaTeX, Microsoft Office}
\cvitem{Video}{Adobe Premiere Pro}

\subsection{Langues}
\cvlanguage{Anglais}{C1-C2}{}
\cvlanguage{Français}{}{}
\cvlanguage{Espagnol}{Débutant}{}

\end{document}