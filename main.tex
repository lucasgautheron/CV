    
\AtBeginDocument{\hypersetup{pdfborder = 0 1 1, linkcolor=blue}}
\documentclass[11pt,a4paper]{moderncv}
\moderncvtheme[blue]{classic}
\AtBeginDocument{\recomputelengths}
\usepackage[utf8]{inputenc}
\usepackage[top=1.1cm, bottom=1.1cm, left=2cm, right=2cm]{geometry}
\setlength{\hintscolumnwidth}{2.5cm}

\RequirePackage{filecontents}
\usepackage[backend=biber,maxbibnames=99,defernumbers=true,sorting=ydnt,giveninits=true]{biblatex}

\firstname{Lucas}
\familyname{Gautheron}
%\title{Etudiant en Physique}              
\address{26bis Rue Damesme}{75013 Paris}    
\email{lucas.gautheron@gmail.com}                      
\homepage{sciencestechniques.fr}
\mobile{06 79 23 86 47} 
\extrainfo{29/01/1995}
%\photo[70pt][0.5pt]{photo.png}



\begin{filecontents*}{\jobname.bib}

@inproceedings{gautheron:hal-03047153,
  TITLE = {{Longform recordings : Opportunities and challenges}},
  AUTHOR = {Gautheron, Lucas and Lavechin, Marvin and Riad, Rachid and Scaff, Camila and Cristia, Alejandrina},
  URL = {https://hal.archives-ouvertes.fr/hal-03047153},
  BOOKTITLE = {{LIFT 2020 - 2{\`e}mes journ{\'e}es scientifiques du Groupement de Recherche ''Linguistique informatique, formelle et de terrain''}},
  ADDRESS = {Montrouge / Virtual, France},
  EDITOR = {Poibeau, Thierry and Parmentier, Yannick and Schang, Emmanuel},
  PUBLISHER = {{CNRS}},
  PAGES = {64-71},
  YEAR = {2020},
  MONTH = Dec,
  KEYWORDS = {daylong recordings ; ecological validity ; automatic speech processing},
  HAL_ID = {hal-03047153},
  HAL_VERSION = {v1},
}

@article{Lavechin2021,
  doi = {10.31234/osf.io/pt9xq},
  url = {https://doi.org/10.31234/osf.io/pt9xq},
  note = {\url{https://doi.org/10.31234/osf.io/pt9xq}},
  year = {2021},
  month = mar,
  publisher = {Center for Open Science},
  author = {Marvin Lavechin and Maureen de Seyssel and Lucas Gautheron and Emmanuel Dupoux and Alejandrina Cristia},
  title = {Reverse-engineering language acquisition with child-centered long-form recordings},
  journal = {Annual Reviews of Linguistics}
}

@article{gautheron_rochat_cristia_2021,
 title={Managing, storing, and sharing long-form recordings and their annotations},
 url={https://psyarxiv.com/w8trm},
 note = {\url{https://psyarxiv.com/w8trm}},
 DOI={10.31234/osf.io/w8trm},
 publisher={PsyArXiv},
 author={Gautheron, Lucas and Rochat, Nicolas and Cristia, Alejandrina},
 year={2021},
 month={May},
 journal={Submitted to \emph{Language Resources and Evaluation}}
}
\end{filecontents*}

\def\makenamesetup{%
  \def\bibnamedelima{~}%
  \def\bibnamedelimb{ }%
  \def\bibnamedelimc{ }%
  \def\bibnamedelimd{ }%
  \def\bibnamedelimi{ }%
  \def\bibinitperiod{.}%
  \def\bibinitdelim{~}%
  \def\bibinithyphendelim{.-}}    
\newcommand*{\makename}[2]{\begingroup\makenamesetup\xdef#1{#2}\endgroup}

\newcommand*{\boldname}[3]{%
  \def\lastname{#1}%
  \def\firstname{#2}%
  \def\firstinit{#3}}
\boldname{}{}{}

% Patch new definitions
\renewcommand{\mkbibnamegiven}[1]{%
  \ifboolexpr{ ( test {\ifdefequal{\firstname}{\namepartgiven}} or test {\ifdefequal{\firstinit}{\namepartgiven}} ) and test {\ifdefequal{\lastname}{\namepartfamily}} }
  {\mkbibbold{#1}}{#1}%
}

\renewcommand{\mkbibnamefamily}[1]{%
  \ifboolexpr{ ( test {\ifdefequal{\firstname}{\namepartgiven}} or test {\ifdefequal{\firstinit}{\namepartgiven}} ) and test {\ifdefequal{\lastname}{\namepartfamily}} }
  {\mkbibbold{#1}}{#1}%
}

\addbibresource{\jobname.bib}
\boldname{Gautheron}{Lucas}{L.}

\begin{document}
\maketitle

\section{Formation}
\cventry{2021-2022}{M2}{Université de Paris}{Paris (75)}{Histoire et Philosophie des Sciences [\textit{En cours}]}{}
\cventry{2014-2018}{L3/M1}{Ecole Normale Supérieure de Cachan}{Cachan (94)}{Physique Théorique et Expérimentale (Phytem)}{Fonctionnaire stagiaire. Options Symétries et intégrales de chemin, Astrophysique et astroparticules.}
\cventry{2012-2014}{PCSI/PC*}{Lycée Berthollet}{Annecy (74)}{}{Classe préparatoire aux grandes écoles}
\cventry{2009 -- 2012}{Baccalauréat S option SI}{Lycée Louis Lachenal}{Argonay (74)}{}{}

\section{Recherche}

\cventry{Septembre 2020 à Novembre 2021}{Ingénieur d'études}{Laboratoire de Sciences Cognitives et Psycholinguistique (LSCP - DEC - École Normale Supérieure)}{}{Paris}{
Étude de l'acquisition du langage à travers les cultures à l'aide d'enregistrements audio de longue durée.
\begin{itemize}%
\item Développement de procédures et d'un package python pour la gestion, le stockage et l'analyse de datasets volumineux ($\mathcal{O}(10^4)$ heures d'enregistrements audio + bases de données et annotations)
\item Traitement du signal sur les enregistrements longs
\item Analyse de données
\item Formation CNRS "Fondamentaux du Machine Learning et du Deep Learning" (28h); apprentissage supervisé et non supervisé, réseaux de neurones; (scikit-learn, keras/tensorflow, pytorch)
\end{itemize}
}

\cventry{Octobre 2016 -- Janvier 2017}{Stage de recherche}{Laboratoire Univers et Théories (LUTH - INSU - CNRS)}{}{Paris Meudon}{
Influence de la distribution complète de noyaux pendant une supernova à effondrement de coeur sur les taux de capture électronique et la diffusion de neutrinos. Supervisé par Micaela Oertel.
\begin{itemize}%
\item Calcul de taux de capture électronique et les sections efficaces de diffusion de neutrinos durant une supernova
\item Integration de ces résultats dans une simulation du processus de supernova (Fortran, C++).
\end{itemize}
}

\cventry{Mai 2016 -- Juillet 2016}{Stage de recherche}{Laboratoire de Physique Nucléaire et des Hautes Énergies (LPNHE - IN2P3 - CNRS)}{}{Paris}{
Analyse diphoton et phénoménologie pour l'expérience ATLAS. Supervisé par Lydia Roos.
\begin{itemize}%
\item Nouvelle paramétrisation pour un signal de désintégration de spin-2 dans la distribution de masse invariante diphoton, en vue d'ICHEP 2016 (Pythia, ROOT, RooFit, C++, Python, FeynRules, CalcHEP)
\item Corrections NLO pour le signal spin-2 (MadGraph5\_aMC\_atNLO)
\end{itemize}
}

\cventry{Octobre 2015 -- Janvier 2016}{Stage de recherche}{Laboratoire d'Annecy-Le-Vieux de Physique Théorique (LAPth - IN2P3 - CNRS)}{}{Annecy-Le-Vieux}{
Cosmologie. Supervisé par Richard Taillet.
\begin{itemize}%
\item Création d'une site Internet pédagogique sur l'histoire de la cosmologie moderne (\href{http://cosmology.education/}{http://cosmology.education/})
\item Développement de plusieurs simulations (en C++) pour l'animation du site
\end{itemize}
}

\cventry{Mai 2015 -- Juillet 2015}{Stage de recherche}{Laboratoire d'Annecy-Le-Vieux de Physique Particules (LAPP - IN2P3 - CNRS)}{}{Annecy-Le-Vieux}{
Physique des particules pour l'expérience ATLAS. Supervisé par Stéphane Jézéquel.
\begin{itemize}%
\item Analyse des événements diphoton et calculs simplifiés de significance locale et globale avec ROOT
\item Anaylse des performances d'une prototype de nouveau tracker en vue d'HL-LHC à l'aide de simulations MC
\item Développement d'une simulation pour estimer l'impact du rayonnement thermique sur la température d'un module pixel du tracker dans le cadre du design du système de refroidissement
\item Développement d'une simulation calculant les intersections de particules chargées dans le tracker avec les capteurs pixels pour un prototype de tracker
\end{itemize}
}

\nocite{*} % <==========================================================
%\bibliographystyle{apalike}
%\bibliography{\jobname} % To use bib file created by filecontents
\printbibliography[title={Publications}]

\section{Journalisme}

%\cventry{Novembre 2020 à Aujourd'hui}{Pigiste}{Le Média}{}{Montreuil}{Recensions d'ouvrages.}

\cventry{Décembre 2019 à Novembre 2020}{Président, directeur de la publication}{Société de Production Le Média}{}{Montreuil}{Administration d'une société de production audiovisuelle (web-télé) disposant de plus de 12 équivalents temps plein. Optimisation des procédures de démarchage des client, analyse de données à des fins de ciblage, développement d'outils de gestion et de prévision}

\cventry{Décembre 2019 à Septembre 2020}{Journaliste (CDI)}{Le Média}{}{Montreuil}{Spécialisation en analyse/représentation de données et recensions d'ouvrages; chef d'édition. \url{https://www.lemediatv.fr/auteurs/lucas-gautheron-9DAnWoo5Tlav1trWgg_Qlw/articles}. Expérience audiovisuelle en montage, cadre, réalisation en direct.
}

\cventry{Septembre 2018 -- Août 2019}{Responsable Numérique}{Le Média}{}{Montreuil}{
\begin{itemize}
\item Gestion de la diffusion des contenus sur les réseaux sociaux : éditorialisation, montage, sous-titrage, programmation. Analyse des audiences.
\item Production (cadrage, réalisation en direct)
\item Post-production (montage)
\end{itemize}
}

\section{Développement}

\cventry{Juillet 2013}{Développeur}{Électricité réseau Distribution de France (ErDF)}{Annecy}{}{Participation au développement d'une application pour l'organisation du travail des salariés
\begin{itemize}%
\item Implémentation d'un système d'archive (PHP/MySQL).
\item Automatisation de l'accès aux données depuis plusieurs applications externes (cURL). \newline{}
\end{itemize}}
\cventry{Mars 2011 à 2014}{Développeur}{AssaultCube}{}{}{Participation bénévole au développement d'un jeu vidéo au sein d'une équipe internationale (C++).}

\section{Compétences}

\subsection{Informatique}
\cvitem{Programmation}{Python, C, C++, Fortran}
\cvitem{Logiciels scientifiques}{numpy, scipy, scikit-learn, MadGraph5\_aMC\_@NLO, Pythia, ROOT/RooFit, stan}
\cvitem{Données}{Pandas, MySQL, HDF}
\cvitem{Web}{PHP, HTML, JS, CSS}
\cvitem{Video}{Adobe Premiere Pro}

\subsection{Langues}
\cvlanguage{Anglais}{C1-C2}{}
\cvlanguage{Français}{}{}
\cvlanguage{Espagnol}{Débutant}{}

\end{document}